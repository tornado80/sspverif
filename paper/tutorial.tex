\newif\iffull\fullfalse% do not change flags here
\newif\ifdraft\draftfalse% do not change flags here

\fulltrue  % full  = long version (stuff to move to appendix)
\drafttrue % draft = comments and half-baked bits

%\pdfsuppressptexinfo-1

%
%\documentclass[letterpaper,table,dvipsnames,10pt]{article}
\documentclass[hyperref,envcountsame,envcountsect,runningheads]{llncs}%
\usepackage[dvipsnames]{xcolor}
\usepackage{hyperref}
% \pagestyle{plain}
%   \addtolength{\oddsidemargin}{-.5in}
%   \addtolength{\evensidemargin}{-.5in}
%   \addtolength{\textwidth}{1in}
\hypersetup{
    bookmarks=true,         % show bookmarks bar?
    unicode=false,          % non-Latin characters in Acrobat�s bookmarks
    colorlinks=true,       % false: boxed links; true: colored links
    linkcolor=RoyalPurple,          % color of internal links (change box color with linkbordercolor)
    citecolor=black,
    filecolor=black,
    urlcolor=black,
    hypertexnames=false
}
 \pagestyle{plain}
%   \addtolength{\oddsidemargin}{-1.6in}
%   \addtolength{\evensidemargin}{-1.6in}
%   \addtolength{\textwidth}{3.2in}

\def\topfraction{0.8}

\setcounter{tocdepth}{2}

%\usepackage{savetrees}
\usepackage{xspace}
\usepackage{lmodern}
\usepackage[utf8]{inputenc}
\usepackage[T1]{fontenc}
\usepackage[english]{babel}
\usepackage{latexsym}
\usepackage{amsmath,amsfonts,amssymb}
\usepackage{nameref}
%\usepackage{cleveref}
%\usepackage{amsthm}
\usepackage{tikz}
\usetikzlibrary{fit}

\usepackage{verbatim}
\usepackage{enumitem}
\usepackage[probability,adversary,sets,operators,primitives]{cryptocode}[=2018-11-11]
\usepackage{wrapfig}
\usepackage{booktabs}
\usepackage{pdfpages}
\usepackage{enumitem}
\usepackage{multicol}
\usepackage{catchfilebetweentags}
% \usepackage{lipsum}
\usepackage{url}
\usepackage{lscape}
\usepackage{subcaption}
\usepackage{todonotes}
\usepackage{placeins}
\usepackage{rotating}
\usepackage{thm-restate}

%\makeatletter
%\let\claim\@undefined
%\let\endclaim\@undefined
%\makeatother
%\spnewtheorem{claim}[theorem]{Claim}{\bfseries}{\itshape}

% include code in verbatim
\usepackage{listings}
\lstset{basicstyle=\footnotesize\ttfamily,
breaklines=true,
numbers=left,
tabsize=4,
escapeinside={<@}{@>}}

\let\llncssubparagraph\subparagraph
\let\subparagraph\paragraph
\usepackage{titlesec}
%\let\subparagraph\llncssubparagraph
\let\paragraph\subsubsection
\titlespacing*{\section} {0pt}{3ex plus 1ex minus .2ex}{2.1ex plus .2ex}
\titlespacing*{\subsection} {0pt}{2.75ex plus 1ex minus .2ex}{1.3ex plus .2ex}
\titlespacing*{\subsubsection}{0pt}{1ex plus 1ex minus .2ex}{2mm}

\newcommand{\ADM}{\text{\textrm{\upshape ADM}}}
\newcommand{\StyleModel}[1]{\V{{#1}}}
\newcommand{\StyleKey}[1]{\ensuremath{\mathit{#1}}}

\renewcommand{\lll}{\text{left}}
\newcommand{\rrr}{\text{right}}
\newcommand{\mmm}{\text{middle}}
\newcommand{\XPN}{\n{XPN}}
\newcommand{\XTN}{\n{XTN}}


\newcommand{\hon}{\V{{hon}}}
\newcommand\admid{\n{admid}}
\newcommand\idx{\O{idx}}
\newcommand\ctr{\n{ctr}}
\newcommand\ks{\n{ks}}

\newcommand{\ksc}{\mathsf{ksc}}
\renewcommand{\vec}[1]{\textbf{#1}}

\newcommand{\outputof}{\mathsf{output}}
\newcommand{\vecgets}{\stackrel{\textbf{vec}}{\gets}}
\newcommand{\choosegets}{\stackrel{\text{choose}}{\gets}}
\newcommand{\cequiv}{\stackrel{\textbf{code}}{\equiv}}
\newcommand{\vecequal}{\stackrel{\textbf{vec}}{=}}


\newcommand{\GKG}{\M{GKG}}
\newcommand{\cnt}{\textsf{cnt}}
\renewcommand{\H}{\textsf{H}}
\newcommand{\RO}{\textsf{RO}}
\newcommand{\RPO}{\textsf{RPO}}
\newcommand{\E}{\textsf{E}}
\newcommand{\ininterface}[1]{{[\M{#1}\rightarrow]}} %{I^{in}_{#1}}
\newcommand{\outinterface}[1]{{[\rightarrow\M{#1}]}} %{I^{out}_{#1}}
\newcommand{\V}[1]{\ensuremath{\mathit{#1}}}

\newcommand{\args}{\StyleModel{args}}
\newcommand{\history}{\StyleModel{history}}
\newcommand{\Handles}{\vec h}
\newcommand{\Keys}{\vec k}
\renewcommand{\hash}{\O{hash}}
\newcommand{\xtr}{\O{xtr}}
\renewcommand{\dh}{\O{dh}}
\newcommand{\hmac}{\O{xtr}}
\renewcommand{\prf}{\hmac}
\newcommand{\xpd}{\O{xpd}}
\newcommand{\hkdf}{\O{xpd}}
\newcommand{\dkdf}{\O{kdf}}
\newcommand{\kdf}{\hkdf}
\newcommand{\gendh}{\O{gendh}}
\newcommand{\honest}{\mathsf{honest}}
\newcommand{\honestShares}{\mathsf{honestShares}}

\newcommand{\ReturnHandles}{\StyleModel{ReturnHandles}}
\newcommand{\CompHand}{\O{CompHand}}
\newcommand{\CompInput}{\O{CompInput}}
\newcommand{\CompKey}{\O{CompKey}}
\newcommand{\compK}[1]{\mathsf{nextKeys}(#1)}
\newcommand{\CompKeys}{\CompKey}
\newcommand{\lastShares}{\mathsf{shares}}
\newcommand{\lastPsk}{\mathsf{psk}}
\newcommand{\level}{\mathsf{level}}
\newcommand{\sort}{\mathsf{sort}}
\newcommand{\length}{\mathsf{len}}
\newcommand{\outlen}{\length}
\newcommand{\keylen}{\length}
\newcommand{\esalt}{\StyleModel{esalt}}
\newcommand{\share}{\StyleModel{share}}
\newcommand{\localshare}{X}% or \share_\local
\newcommand{\peershare}{Y}% or \share_\peer
\newcommand{\exponent}{\StyleModel{exp}}
\newcommand{\shares}{\StyleModel{shares}}
\renewcommand{\th}{\StyleModel{th}}
%\renewcommand{\dh}{\StyleModel{dh}}
\newcommand{\hs}{\StyleModel{hs}}
\newcommand{\bind}{\StyleModel{bind}}
\newcommand{\eem}{\StyleModel{eem}}
\newcommand{\es}{\StyleModel{es}}
\newcommand{\cet}{\StyleModel{cet}}
\newcommand{\cekey}{\StyleModel{cekey}}
\newcommand{\ceiv}{\StyleModel{ceiv}}
\newcommand{\Label}{\lbl{label}}
\newcommand{\ctx}{\V{ctx}}
\newcommand{\isres}{\StyleModel{is}\_\StyleModel{res}}

\newcommand{\stack}[1]{{\scriptsize\begin{array}{l} #1 \end{array}}}
\newcommand*{\stacktwo}[2]{\genfrac{}{}{0pt}{}{#1}{#2}}

%\newcommand{\name}{\O{name}}
\newcommand{\type}{\mathsf{type}}
\newcommand{\OutIndexes}{Fin \cup Out}

\newcommand{\pcvar}[1]{\ensuremath{\mathit{#1}}}



\newcommand{\GNACR}{\M{GCR}}
\newcommand{\GCR}{\M{GCR}}
\newcommand{\GCRLIB}{\M{GACR}}
\newcommand{\dual}[1]{#1^{\dagger}}
\newcommand{\falgo}{\V{f}}
\newcommand{\family}{\V{afun}}
\newcommand{\sha}{\O{sha}}
\newcommand{\dash}{\text{-}}

\newcommand{\GPR}{\M{GPR}}
\newcommand{\GPI}{\M{GPI}}
\newcommand{\DKDF}{\M{DKDF}}
\newcommand{\DPRF}{\M{DPRF}}
\newcommand{\TKDF}{\M{TKDF}}
\newcommand{\GDKDF}{\M{GDKDF}}
\newcommand{\GMAC}{\M{GMAC}}
\newcommand{\GXPD}{\M{GXPD}}
\newcommand{\GKDF}{\M{GXPD}}
\newcommand{\GSODH}{\M{GSODH}}
\newcommand{\GMOSODH}{\M{GSODH2}}
\newcommand{\GMOSODHT}{\M{GSODH3}}

\newcommand{\XPD}{\M{XPD}}



\newcommand{\KS}{\M{KS}}
\newcommand{\GKS}{\M{GKS}}
\newcommand{\GMAIN}{\M{GMAIN}}
\newcommand{\GCORE}{\M{GCORE}}
\newcommand{\RDDF}{\V{RD,DF,DD}}


\newcommand{\CORE}{\M{CORE}}
\newcommand{\CONTROL}{\M{CONTROL}}
\newcommand{\eph}{\StyleModel{eph}}
\newcommand{\stat}{\StyleModel{stat}}
\newcommand{\kgen}{\StyleModel{kgen}}
\newcommand{\m}{\StyleModel{m}}
\renewcommand{\mac}{\O{mac}}
\newcommand{\dprf}{\StyleModel{dprf}}
\renewcommand{\enc}{\StyleModel{enc}}
\renewcommand{\dec}{\StyleModel{dec}}
\newcommand{\send}{\StyleModel{send}}
\newcommand{\rec}{\StyleModel{rec}}
\renewcommand{\verify}{\StyleModel{verify}}
\newcommand{\se}{\StyleModel{se}}
\newcommand{\ch}{\StyleModel{ch}}
\renewcommand{\O}[1]{\ensuremath{\mathsf{#1}}}
\newcommand{\M}[1]{\ensuremath{\text{\textrm{\upshape#1}}}}
\newcommand{\G}{\M{G}}
\newcommand{\unfcma}{\M{UNF-CMA}}
\newcommand{\indcca}{\M{IND-CCA}}
\newcommand{\indcpa}{\M{IND-CPA}}
\newcommand{\dollarindcpa}{\M{\$-IND-CPA}}
\newcommand{\dollarindcca}{\M{\$-IND-CCA}}
\newcommand{\pcassert}{\mathbf{assert}\;}
\newcommand{\pcand}{\mathbf{and}\;}
\newcommand{\pcwhere}{\mathbf{where}\;}
\renewcommand{\ae}{\M{AE}}
\newcommand{\dollarae}{\M{\$-AE}}
\newcommand{\pctwo}[2]{\frac{#1}{#2}}
\newcommand{\advantage}{\text{Adv}}
\newcommand{\eps}{\mathsf{Adv}}
\newcommand{\seq}{\circ}
\newcommand{\sequiv}{\stackrel{s}{\approx}}
%\newcommand{\cequiv}{\stackrel{\text{code}}{\equiv}}
\newcommand{\oequiv}{\stackrel{o}{\approx}}
\newcommand{\pctwoh}[2]{\left(#1 \middle| #2\right)}
\newcommand{\ID}{\M{ID}}
\newcommand{\KEY}{\M{KEY}}
\newcommand{\SKEY}{\M{K}\xspace}
\newcommand{\LOG}{\M{LOG}\xspace}
\newcommand{\SLOG}{\M{L}\xspace}
\newcommand{\pat}{\V{pat}}
\newcommand{\TRANS}{\M{TRANS}}
\newcommand{\XPR}{\V{XPR}}
%\newcommand{\keylen}{\lambda} %\StyleModel{keylen}}
\newcommand{\GEN}{\O{GEN}}
\newcommand{\SET}{\O{SET}}
\newcommand{\CSET}{\O{CSET}}
\newcommand{\GET}{\O{GET}}
\newcommand{\HON}{\O{HON}}
\newcommand{\UNQ}{\O{UNQ}}
\newcommand{\HASH}{\O{HASH}}
\newcommand{\mi}[1]{\prod_{i=1}^{n} #1[i]}
\newcommand{\miminus}[1]{\prod_{i=1}^{n} #1[i-]}
\newcommand{\mij}[1]{\prod_{j=1}^{n} #1[j]}
\newcommand{\mijNoSub}[1]{\prod_{j=1}^{n} {#1}}
\newcommand{\miSpecific}[3]{\prod_{i=#2}^{#3} #1[i]}
\newcommand{\miSpecificj}[3]{\prod_{j=#2}^{#3} #1[j]}
\newcommand{\pk}{\StyleKey{pk}}
\newcommand{\ppk}{\StyleKey{ppk}}
\newcommand{\sk}{\StyleKey{sk}}
\newcommand{\running}{\StyleModel{running}}
\newcommand{\rejected}{\StyleModel{rejected}}
\newcommand{\initiator}{\StyleModel{initiator}}
\newcommand{\responder}{\StyleModel{responder}}
\newcommand{\accepted}{\StyleModel{accepted}}
\newcommand{\sid}{\StyleModel{sid}}
\newcommand{\master}{\StyleModel{master}}
\newcommand{\Out}{\StyleModel{Out}}
\newcommand{\static}{\mathsf{static}}

% Superscripts for relational proofs
\renewcommand{\r}[1]{#1^{\mathit{r}}}
\renewcommand{\l}[1]{#1^{\mathit{l}}}
\newcommand{\lpar}[1]{(#1)^{\mathit{l}}}
\newcommand{\rpar}[1]{(#1)^{\mathit{r}}}

\newcommand{\corrupted}{\StyleModel{corrupted}}
\newcommand{\run}{\StyleModel{run}}
\newcommand{\state}{\StyleModel{state}}

\newcommand{\A}{\mathcal{A}}

\newcommand{\dom}{\mathop{dom}}
\newcommand{\pcabortwith}[1]{\textbf{throw }\constn{#1}}
\newcommand{\pcmemorize}{\textbf{memorize}}
\newcommand{\pconce}{\textbf{once per handle}}

% See examples below for usage.
% Please suggest better macro names!
\newcommand{\name}{\O{name}}
\newcommand{\fullname}{\O{fullname}}
\newcommand{\n}[1]{\V{#1}}
\newcommand{\constn}[1]{\mathtt{#1}}
\newcommand{\nname}{\n{n}}
\newcommand{\nem}{\n{eam}}
\renewcommand{\index}{{\n{n}}}
\newcommand{\pindex}{{{\n{n}\text{-}}}}
\newcommand{\h}[1]{h_{\n{#1}}}
\renewcommand{\k}[1]{k_{\n{#1}}}
\newcommand{\trs}[1]{\mathit{t}_{\n{#1}}}
\newcommand{\dig}[1]{\mathit{d}_{\n{#1}}}
\newcommand{\lbl}[1]{{\mbox{\texttt{#1}}}}% omit the quotes?
\newcommand{\node}[2]{\mathsf{#1}\langle{#2}\rangle} % constructor, parent name, other args
\newcommand{\expandtr}[2]{\node{xpd}{\h{#1},\lbl{#2},\trs{#1}}}
\newcommand{\expand}[2]{\node{xpd}{\h{#1},\lbl{#2}, []}} %\lbl{},
\newcommand{\extract}[1]{\node{xtr}{\node{noSALT}{\alg{\h{#1}}},\h{#1}}}
\newcommand{\extractdh}[2]{\node{xtr}{\h{#1},#2}}
\newcommand{\extracta}[1]{\node{xtr}{\h{#1},\node{noIKM}{\alg{\h{#1}}}}}
\newcommand{\nodeCTL}[3]{\mathsf{#1}_\n{#2,\level(\h{#2})}({#3})} % constructor, parent name, other args
\newcommand{\expandtrCTL}[2]{\O{EXPAND}_\n{#1,\level(\h{#1})}(\h{#1},\lbl{#2},\trs{#1})}
\newcommand{\expandCTL}[2]{\O{EXPAND}_\n{#1,\level(\h{#1})}( \h{#1},\lbl{#2}, [] )} %\lbl{},
\newcommand{\extractCTL}[1]{\O{EXTRACT}_\n{#1,\level(\h{#1})}(\node{noSALT}{\alg{\h{#1}}},\h{#1}))}
\newcommand{\extractdhCTL}[2]{\O{EXTRACT}_\n{#1,\level(\h{#1})}(\h{#1},#2)}
\newcommand{\extractaCTL}[1]{\O{EXTRACT}_\n{#1,\level(\h{#1})}(\h{#1},\node{noIKM}{\alg{\h{#1}}})}
\newcommand{\KDF}[3]{\mathsf{xpd}(\k{#1},\lbl{#2},#3)} %used to be hkdf
\newcommand{\compH}[1]{\mathsf{nextHandle}(#1)}
\newcommand{\compCTL}[1]{\mathsf{DERIVE}(#1)}
\newcommand{\local}{\V{local}}
\newcommand{\peer}{\V{peer}}

\newcommand{\tagkey}[2]{\O{tag}_{#2}(#1)}
\newcommand{\untagkey}[2]{\O{untag}(#1)} %removed _{#2}
\newcommand{\cbuntagkey}[1]{\O{untag}(#1)} %cb: a macro that drops one argument is terrible, defined my own...
\newcommand{\alg}[1]{\O{alg}(#1)}
\newcommand{\algo}{\V{alg}}
\newcommand{\leng}{\V{len}}
\newcommand{\len}{\O{len}}
\newcommand{\hashAlgs}{{\cal H}}
\newcommand{\format}[1]{\O{format}(#1)}
\newcommand{\Formats}{\V{Formats}}
\newcommand{\Groups}{\mathcal{G}}
\newcommand{\generator}{\O{gentor}}
\newcommand{\gen}{\O{gen}}
\newcommand{\group}{\O{grp}}
\newcommand{\grp}{\V{grp}}
\newcommand{\order}{\O{ord}}


\newcommand{\lev}[1]{\O{level}(#1)}
\newcommand{\res}[1]{\O{res}(#1)}
\newcommand{\key}[1]{\O{key(#1)}}
\newcommand{\truncate}{\O{truncate}}
\newcommand{\binderTag}{\O{binderTag}}
\newcommand{\binderHandle}{\O{binderHandle}}

\newcommand{\isresflag}{\V{r}}
\newcommand{\mactag}{\V{mac}}

\newcommand{\lname}{\n{ln}}
\newcommand{\rname}{\n{rn}}
\newcommand{\oname}{\n{on}}
\newcommand{\poname}{\n{on}\text{-}}
\newcommand{\pOnames}{\n{M}\text{-}}
\newcommand{\Onames}{\n{M}}
\newcommand{\iname}{\n{in}}
\newcommand{\sname}{\n{sn}}

\newcommand{\mname}{\n{m}}
\newcommand{\ESet}{M\cup N}
\newcommand{\event}{e}
\newcommand{\rdvMul}{\adv_{\V{c}}}
\newcommand{\rdvMaxi}{\adv_{\V{i}}}
\newcommand{\map}{\V{map}}
\newcommand{\hyb}{\M{HYB}}
\newcommand{\hpr}{\M{HPR}}
\newcommand{\hkg}{\M{HKG}}

\newcommand{\GDPRF}{\M{GDPRF}}
\newcommand{\GDPRFpr}{\M{GDPRF-LPR}}
\newcommand{\GDPRFrpr}{\M{GDPRF-RPR}}
\newcommand{\GDPRFkg}{\M{GDPRF-LKG}}
\newcommand{\GDPRFrkg}{\M{GDPRF-RKG}}
\newcommand{\GPRLib}{\M{GPRLIB}}
\newcommand{\GKGLib}{\M{GKGLIB}}

\newcommand{\GPRL}{\M{GPRLIB}_\lname}
\newcommand{\GPRR}{\M{GPRLIB}_\rname}
\newcommand{\GKGL}{\M{GKGLIB}_\lname}
\newcommand{\GKGR}{\M{GKGLIB}_\rname}
\newcommand{\EVALL}{\O{EVAL}_\lname}
\newcommand{\EVALR}{\O{EVAL}_\rname}

\newcommand{\compose}{\text{\textemdash}}

\newcommand*{\pcthree}[3]{%
  \begin{array}{@{\,}c@{\,}}%
    #1\\
    \hline
    #2\\
    \hline
    #3%
  \end{array}%
}

\newcommand*{\pcfour}[4]{%
  \begin{array}{@{\,}c@{\,}}%
    #1\\
    \hline
    #2\\
    \hline
    #3\\
    \hline
    #4%
  \end{array}%
}


\newcommand*{\pcfive}[5]{%
  \begin{array}{@{\,}c@{\,}}%
    #1\\
    \hline
    #2\\
    \hline
    #3\\
    \hline
    #4\\
    \hline
    #5%
  \end{array}%
}


\newcommand\CND[2]{\ensuremath{\mathit{(#1}\text-\mathit{#2)}}}
\newcommand\Package[1]{\hyperref[code:#1]{$\M{#1}$}}

% \mathsf{KDF}(\h{#1}.alg,

%%% Local Variables:
%%% mode: latex
%%% TeX-master: "main"
%%% End:


%\usepackage{ntheorem}
%\theoremstyle{definition}
%\newtheorem{definition}{Definition}[section]
% \newtheorem{remark}{Remark}
%\newtheorem{corollary}{Corollary}
%\newtheorem{conjecture}{Conjecture}
%\newtheorem{construction}{Construction}
%\newtheorem{example}{Example}
%\newtheorem{hint}{Hint}
%\newtheorem{theorem}{Theorem}
%\newtheorem*{lemma*}{Lemma}
%\newtheorem{sublemma}{Lemma}[lemma]
%\newtheorem{myclaim}{Claim}[lemma]

\newcommand\defeq{\mathrel{\overset{\makebox[0pt]{\mbox{\normalfont\tiny\sffamily
def}}}{=}}}
\def\checkmark{\tikz\fill[scale=0.4](0,.35) -- (.25,0) -- (1,.7) --
  (.25,.15) -- cycle;}

% Submission version
%\newcommand{\workingnotes}[1]{}
%\newcommand{\workingnotes}[1]{\marginpar{\scriptsize{#1}}}
%\newcommand{\workingnotes}[1]{{#1}}

\newcommand{\vcaption}[1]{\vspace{-1ex}\caption{#1}\vspace{-1ex}}
\newcommand{\maxres}{L}
\newcommand{\psk}{\O{psk}}
\newcommand{\afterdeadline}[1]{}
\setuptodonotes{shadow}
\newcommand{\adl}[1]{} %{\todo[color=lightgray]{ADL: #1}}
\newcommand{\cb}[1]{} %{\todo[color=lime]{CB: #1}}
\newcommand{\chris}[1]{} %{\todo[color=lime]{CB: #1}}
\newcommand{\kk}[1]{} %{\todo[color=green!30]{KK: #1}}
\newcommand{\cf}[1]{} %{\todo[color=blue!30]{CF: #1}}
\newcommand{\CF}[1]{} %{\todo[color=blue!30,bordercolor=red]{CF: #1}}
\newcommand{\mk}[1]{\todo[color=red!30]{MK: #1}}
\newcommand{\MK}[1]{\todo[color=red!30,bordercolor=red]{MK: #1}}
\newcommand{\mklight}[1]{}
\renewcommand{\vec}[1]{\ensuremath{\mathbf{#1}}}
\renewcommand\pcthen{:}

\title{SSBee Tutorial}
%\author{}\institute{}
\author{Chris Brzuska\inst{1}, Christoph Egger\inst{2}, Jan Winkelmann\inst{3}}
\institute{
			 Aalto University, Finland \and Chalmers University, Sweden \and Free spirit
}

\let\oldprocedure\procedure
\renewcommand\procedure[2][]{\oldprocedure[#1,codesize=\small{}]{\small#2}}

\definecolor{kxpink}{RGB}{211,77,122} % D34D7A
\newcommand\pink{\textcolor{kxpink}{pink}}
\definecolor{kxblue}{RGB}{51, 171, 189} % #33abbd
\newcommand\blue{\textcolor{kxblue}{blue}}
\newcommand\gray{\textcolor{gray}{gray}}


\begin{document}
\maketitle
%\vspace{-1.3cm}


\tableofcontents

\clearpage
\section{Technical overview}\label{sec:overview}
This is a technical overview. See Section~\ref{sec:conceptualoverview} for a \emph{conceptual} overview. In order to run SSBee, you need a recent version of Rust as well as the SMT-solver CVC5 (or the SMT-solver Z3). In example-projects, you find the hello-world folder which is a good starting point to see how SSBee projects are structured. 

\begin{wrapfigure}{R}{0.18\textwidth}
\vspace{-0.7cm}
\scalebox{0.8}{
\includegraphics{hello-world.png}
}
\vspace{-1cm}
\end{wrapfigure}

Namely, the hello-world folder contains packages which contain oracle parameters, package state and oracles operating on the oracle state (cf. Section~\ref{sec:packages}), games which contain game parameters and can wire instances of the previously defined packages into a game (cf. Section~\ref{sec:games}). Finally, the proof folder contains one or multiple proofs which contain one more multiple statements to be proven about previously defined games; a proof step is either a \emph{reduction} or a \emph{code equivalence}. See Section~\ref{sec:proofs} for details. Packages have file extension .pkg.ssp, games have file extension .comp.ssp (for \underline{comp}osition), and proofs just have file extension .ssp. Here are three useful commands which you can run in the hello-world folder (or any other project folder):

\begin{description}
\item[cargo run -p ssbee prove:] This command verifies the proof in the folder.
\item[cargo run -p ssbee prove --transcript:] This command verifies the proof in the folder and writes the resulting smt file into hello-world/$\_$build/code$\_$eq, which is usually not needed, but can be interesting if you want to understand what is going on on the SMT level, especially if the SMT solver gives you a counterexample to your proof.
\item[cargo run -p ssbee latex:] This command generates LaTeX code to nicely display the code and games of your project (cf. Section~\ref{sec:latex}) in a PDF. Before doing so, it type-checks the code of packages and games and verifies the reductions, since the command also generates pictures for reductions. This command does not verify code-equivalences.
\end{description}


\subsection{Packages}\label{sec:packages}

\begin{wrapfigure}{R}{0.18\textwidth}
\vspace{-0.75cm}
\scalebox{0.7}{
\includegraphics{hello-world-rnd.png}
}
\vspace{-1cm}
\end{wrapfigure}

We explain the structure of a package on the example of the $\M{Rand}$ package in hello-world/packages. Conceptually, this package allows to do a (somewhat) useful operation and a pretty useless operation. The useful operation is that it allows to sample random strings from $\bin^n$ and counts how many random strings it has already sampled. In turn, the (pretty) useless operation takes an integer as input and throws an error whenever that integer is not $1$.

\begin{wrapfigure}{R}{0.615\textwidth}
%\vspace{-0.5cm}
\scalebox{0.37}{
\includegraphics{hello-world-rnd-code.png}
}
\vspace{-1cm}
\end{wrapfigure}

Let's now have a look at the actual implementation of the $\M{Rand}$ package. Concretely, this package has $n$ as a parameter (which is needed to define the set $\bin^n$) and maintains a counter {\tt{ctr}} in its state (which is an integer). The $\O{UsefulOracle}$ can be queried with an empty input, then increments the counter {\tt{ctr}}, samples a random string {\tt{rand}} uniformly from the set $\bin^n$ and returns the pair $({\tt{ctr}},{\tt{rand}})$, i.e., an integer and a string of type Bits(n), our encoding of $\bin^n$. The $\O{UselessOracle}$, in turn, can be queried with any integer $x$. If $x\neq 1$, 

\noindent
the $\O{UselessOracle}$ throws an error (This is the meaning of $\mathbf{assert}$.). Else, it samples a random string {\tt{rand}} uniformly from the set $\bin^n$ (which it does not use and thus is a useless operation) and then returns $1$. See Appendix~\ref{app:code} for a complete description of the SSBee pseudo-code syntax.

Additionally, the packages folder also contains the $\M{Fwd}$ package which also has a $\O{UsefulOracle}$, but its code is different. Namely, when called (with an empty input), $\O{UsefulOracle}$ just forwards the call to another $\O{UsefulOracle}$ oracle (provided by some other package) and returns whatever that call returns.
In turn, $\M{Fwd}$ also has a $\O{UselessOracle}$ with the same code as in the $\M{Rand}$ package, except that it omits the useless additional sampling of {\tt{rand}}.


\subsection{Games}\label{sec:games}
We explain the structure of a game on the examples of the games $\M{SmallComposition}$ and $\M{BigComposition}$ in hello-world/games.


%To turn a package into a package instance, the game specifies
%use the packages and finally one or multiple proof files which specify game instances
%If you run cargo run prove in the hello-world folder, SSBee will verify the proof in hello-world (to be discussed shortly). If you run cargo run latex,

\subsection{Proofs}\label{sec:proofs}


\subsection{Generating LaTeX}\label{sec:latex}


%\begin{abstract}
%Very concrete.
%\end{abstract}

%\tableofcontents

%\section{Introduction}
Halevi's dream-vision of computer-aided reduction proofs in cryptography~\cite{Halevi} hopes that one day, while creativity is still reserved for humans, algorithms will be in charge of tedious steps which confirm the soundness of a reduction and only use standard techniques. An important, somewhat orthogonal, success on the path for computer-aided cryptography is that computer-aided \emph{attack-finding} has established itself firmly as a standard tool for protocol development. Most importantly, Tamarin~\cite{X} has been used to improve ...~\cite{X}, ...~\cite{X} and ...~\cite{X}, and, in particular, is used across the community also without involving an author of the original tool, see, e.g.~\cite{X}. However, computer-aided \emph{reduction proofs} have not yet joined the quest to inform protocol design quickly. In fact, to date, reduction proofs for large protocols such as TLS and MLS have not been computer-verified, let alone been helped by algorithms, and indeed, we do not yet have the tools to do so.

One core reason is that invoking the help of a computer requires formalizing one's proofs at a level of detail which is unusual for standard mathematical practice, including cryptography. For example, one might describe two security games and a construction in English, then describe a reduction in English and prove its soundness by arguing about difficult key points. In their celebrated foundational work, Bellare and Rogaway~\cite{X} encouraged the community to use \emph{code} to describe their reductions, games and game-hops and demonstrated how code-based argument allow us to obtain a neat proof for ... . This leap towards more precision enabled ... ... and ...~\cite{X} to develop the pioneer tool EasyCrypt for verifying cryptographic reductions. EasyCrypt has been used to verify ... ,... and ... as well as a proof of Yao's garbling scheme~\cite{X} as one of its largest case studies. Simultaneously, Bellare-Rogaway-style game-hopping has become an established technique in our community. Similarly, the composition frameworks Universal Composability (UC~\cite{X}) or Constructive Cryptography (CC~\cite{X}) have a large and growing user base who write their reduction proofs in these frameworks (cf. ... and references therein). This suggests that our community is willing to invest time and effort into writing clearer proofs, but the overhead currently required by EasyCrypt is still considered prohibitive. 

A recent arrival to the family of composition frameworks are \emph{state-separating proofs} (SSPs~\cite{X}) where the first large case studies have been published over the past 2 years. Concretely, SSPs have been used to provide reductions for the key derivations in both TLS~\cite{X} and MLS~\cite{X} and, most recently, to give a novel security reduction for Yao's garbling scheme~\cite{X}, demonstrating the potential feasibility to use SSPs also for secure multi-party computation.

Similarly to Bellare and Rogaway's code-based game-playing, SSPs, a refinement of Bellare-Rogaway, are formalization-friendly and have since been formalized in Coq~\cite{X} as well as in EasyCrypt~\cite{X}. In fact, SSPs have become a useful manual prototyping step for developing proofs in EasyCrypt [Dupressoir-Oechsner, personal communication :-)]. What is missing, still, however, is a formal verification tool which \emph{helps} cryptographers in the \emph{development} and \emph{execution} of their proofs and not only in the verification.

\paragraph{Our contribution.} In this paper, we provide such a tool for SSPs: SSPVerif. 
%SSPs have shown to be useful to carrying out proofs for large real-life protocols~\cite{X,Y} and thus, focusing on SSPs is a useful scope. 
We built SSPVerif, because in our development of handwritten SSP proofs for large protocols, we struggled (a) with the \emph{maintainance} of a \emph{large code-base}, (b) with verifying the \emph{consistency} of the code-base and last, but not least, (c) with carrying out code-equivalence proofs. The latter cover ... pages of the security analysis of TLS 1.3~\cite{X} and ... out of ... in the reduction proof for Yao's garbling scheme~\cite{X}. They have been completely avoided in the security reduction for MLS by the use of modular assumptions and modular top-level target security definitions~\cite{X}.

SSPVerif addresses (a) and (b) by checking consistency of packages and their compositions as well as exporting to LaTeX for a consistent research paper. For (c), code-equivalence proofs are discharged to an SMT-solver, either CVC4~\cite{X} or Z3~\cite{X}. Our case studies demonstrate that CVC4 and Z3 are, indeed, sufficiently powerful to prove typical code-equivalences with no to little manual support. 

\paragraph{Case studies}
Yao (compare with EasyCrypt), NPRF

\paragraph{Features}
for loops, multi-instance, type system

\paragraph{Limitations and future work}
SSPVerif is automatized but special-case, EasyCrypt/SSProve is general, but not automatized. Can we use SSPVerif for prototyping of invariants for EasyCrypt/SSProve? Can we use EasyCrypt to prove assumptions used in our SSP proofs. Integration with proof viewer.





\iffalse
It is no co-incidence that EasyCrypt~\cite{X}, the pioneer tool in verifying cryptographic reduction proofs, is based on Bellare and Rogaway's code-based game-hopping~\cite{X}. Namely, Bellare and Rogaway specified 



While computer-aided attack-finding has become a standard tool in protocol development


In a celebrated breakthrough, Abadi and Rogaway~\cite{X} showed that 









==== EVEN OLDER ====


Since Halevi~\cite{X} declared a \emph{crisis of rigour} for reduction proofs in cryptography
and called for tool support to leave the \emph{interesting} steps to the creative cryptographer
and the \emph{mundane} steps to a powerful automated tool, significant progress has been made
towards this vision. In particular, Dupressoir and some others~\cite{X} formalized the Bellare-Rogaway
code-based game-playing~\cite{X} approach in EasyCrypt which now allows to verify code transformations
and reduction-based game hops. 

EasyCrypt does not take care of mundane steps fully automatically, as in Halevi's vision, but rather is semi-automated
and usually requires cryptographers to write pre-conditions and post-conditions (or simply a single invariant) and prove
that if the pre-condition was satisfied before an oracle call, then the post-condition is satsified.
The reason that the SMT solver underlying EasyCrypt cannot prove code equivalence without additional
help is that conceptual high-level reasoning is not part of the solver. The present paper does not
change this state of affairs, but instead increases \emph{modularity} of the EasyCrypt technique.

Concretely, we present SSPVerif, a new verification tool for cryptographic game-hopping proofs which
is based on \emph{state-separating proofs} (SSPs), a recent, modular technique for game-hopping in pen-and-paper
proofs. SSPs structure code of cryptographic protocols and security game as stateful pieces of code
called \emph{packages} which can call one another, but otherwise do not share state. Since package
composition is associative, one can simplify complex game-hops by ``pushing code into the adversary''
(cf. Section~\ref{X}).

SSP-style techniques have already been formalized in EasyCrypt~\cite{X}, but since package are not
native in EasyCrypt, the implementation does not provide SSP-specific support and, in particular,
the SSP-style workflow. SSPs have also been formalized in SSProve~\cite{X}, a Coq-based tool specially
developed for SSPs, but no large examples have been shown in SSProve. In turn, SSPVerif has been developed
to especially support the workflow of the working SSP cryptographer.

As an example, we use Yao's garbled circuits: much larger than what is known in SSProve; Yao has already 
been done on EasyCrypt, so great test case.
\fi
%\section{State-Separating Proofs}
In this section, we review state-separating proofs (SSPs)
as introduced by Brzuska, Delignat-Lavaud, Fournet, Kohbrok and
Kohlweiss (BDEFKK~\cite{X}), integrating refinements and
suggestions proposed in subsequent works relying on SSPs~\cite{blanket-cite,konrads-thesis,sabinesworks}. While reviewing core SSP concepts,
we also explain their encoding in SSBee.

\subsection{Games and Packages}
\paragraph{Games.}
Following Bellare and Rogaway~\cite{Bellare-Rogaway}, a cryptographic game is a set
of oracles operating on shared state.
\begin{definition}
A game $\M{G}$ consists of a set of oracles
which operate on a shared state.
\end{definition}
For example, indistinguishability under chosen plaintext attacks (IND-CPA)
can be formalized as indistinguishability between two games $\M{IND-CPA}^b$, 
each of which provides an encryption oracle $\O{ENC}$ to the adversary that
either encrypts the left or the right input message. The state of 
$\M{IND-CPA}^b$ is only the symmetric encryption key $k$. See Fig.~\ref{X}
for details. Note that search games such as unforgeability under chosen
plaintext attacks (UNF-CMA) of message authentication codes (MACS) and, in general,
any search game can also be encoded as distinguishing games. See Appendix~\ref{app:dist} for details. This is important because SSPs restrict their attention to \emph{distinguishing games}.

\paragraph{Packages.}
If we want to decompose a game into multiple pieces of code, those pieces
of code need to be able to call one another. Therefore, the native citizen
of SSPs is a \emph{package} which generalizes the notion of a game in that
a package not only exposes oracles, but can also \emph{call} oracles (of
other packages, not of itself).

\begin{definition}[Package]
A package $\M{M}$ provides a set of oracles $[\rightarrow \M{M}]$
which operate on a shared state and make calls to a set of oracles
$[\M{M}\rightarrow]$ which we call the \emph{dependencies} of $\M{M}$.
\end{definition}

We can then construct games by modularly describing them as a composition
of several packages.

\paragraph{SSBee.} In SSBee, the user defines each package by specifying
the parameters (e.g., security parameter, encryption schemes), state
variables (e.g., the key $k$ in the IND-CPA game) as well as pseudo-code
for the oracles. The content of variables other than state variables will
be erased in the end of an oracle call. See Fig.~\ref{X} for a description
of the IND-CPA package in SSBee.

\paragraph{Composed games.} In order to specify a (composed) game in SSBee, 
we need to specify the packages that it consists of as well as their wiring,
i.e., which package calls which oracles of which other package as well as
which oracles are exposed to the adversary. For example, let us
decompose the IND-CPA game into two packages, 
a $\M{KEY}$ package (cf. Fig.~\ref{X})
which stores the key $k$ and a stateless 
$\M{ENCRYPT}^b$ package (cf. Fig.~\ref{X}). Now, the composition ...
specifies that $\M{ENCRYPT}^b$ calls the $\O{GET}$ oracle of $\M{KEY}$,
and the adversary can call the $\O{SAMPLE}$ oracle of $\M{KEY}$ and the
$\O{ENC}$ oracle of $\M{ENCRYPT}^b$.

{\color{blue}Chris: Can compositions be parametrized with a bit $b$ and
use different packages based on which value $b$ has?}



%\section{SSBee}
\subsection{Pseudo-code}
We already saw an example of SSBee pseudocode in Section~\ref{sec:ssp} and now discuss the features and limitations of SSBee pseudocode more comprehensively.
In terms of basic operations, SSBee pseudocode supports assignment to state variables and temporary variables (whose value will be forgotten after the oracle call), creating and parsing tuples, sampling uniformly from a set, and evaluation of unspecified function as well as oracle calls. State variables can be individual values or tables. Pseudo-code can contain a special $\mathbf{assert}\mathsf{ cond}$ command which aborts the oracle run if condition $\mathsf{cond}$ is violated. Moreover, SSBee supports if-then-else branching as well as for loops.

\paragraph{Type system.}
SSBee operates on typed symbols, e.g., symbols can be integers (Int),
Booleans (Bool), bitstrings of length n (Bits(n)), tuples of values,
tables etc.. Moreover, the user can specify additional types. For 
every type $\mathsf{type}$, the operation $\mathsf{Maybe}\;\mathsf{type}$
creates a larger type which contains $\mathsf{type}$, but a value
might also be $\mathsf{none}$.

Maybe types are particularly useful for tables. Concretely, at each position, a table might either have a value or be unassigned. Thus, when accessing a table at a position, it is unclear whether it is of the type of values that the table typically contains or of type $\mathsf{none}$. Therefore, a value $v\leftarrow T[i]$ is considered to be of $\mathsf{Maybe}\;\mathsf{type}$, and to turn $v$ into $\mathsf{type}$, we ... which we can only apply if we are guaranteed that $v\leftarrow T[i]$ is not 
equal to $\mathsf{none}$.

\paragraph{Limitations of SSBee pseudocode.}
Sampling from arbitrary distributions needs to be modeled by sampling uniformly from
a set and applying a function to the result. Since SSBee pseudocode is symbolic, it
does not have a concept of length for bitstrings. As a result, in the IND-CPA encryption example, the oracle accepts messages of a \emph{fixed} length $n$ rather
than to allow for different message lengths and checking that the two provided messages are of equal length.

\subsection{Proofs}
As briefly hinted in Section~\ref{X}, SSBee supports \emph{reduction proofs}
as well as \emph{code equivalence proofs}. The user specifies the sequence
of games...example? \emph{reduction proofs}


{\color{blue}Chris: Maybe, the following should move somewhere else, because
it is not SSBee-specific?}
\paragraph{Code equivalence proofs.} Code equivalence proofs between a left
game and a right game proceed by \emph{induction} over the number of oracle calls. We first define a state relation and show that it holds on the pair of initial
states. Next, we need to show the following two statements:
\begin{itemize}
\item[(1)] If the state relation holds before an oracle call, then it holds after an oracle call.
\item[(2)] If the state relation holds before an oracle call, then the two games either both abort or both return the same output.
\end{itemize}
For item (2) (and sometimes also for item (1)), it is necessary to relate the
\emph{randomness} of two games


\paragraph{for-loops.}



\paragraph{Hybrid proofs.} Hybrid proofs are sequences of reduction steps
and code-equivalence proofs, so the SSBee merely ties those together.




%\input{technical-overview}

%\input{notation}
%\input{key-exchange-syntax}
%\input{key-schedule-syntax}




%{\footnotesize
%\bibliography{cryptobib/abbrev3,cryptobib/crypto,misc}{}
%\bibliography{cryptobib/abbrev3,cryptobib/crypto,local}{}
%\bibliographystyle{abbrv}
%}
%\clearpage

\appendix
\section{Code}\label{app:code}

\end{document}
